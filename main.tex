\documentclass[12pt,a4paper]{article}

% Configurações de pacotes
\usepackage[utf8]{inputenc}
\usepackage[portuguese]{babel}
\usepackage{times} % Times New Roman
\usepackage{setspace} % Controle de espaçamento
\usepackage{geometry} % Margens
\usepackage{titlesec} % Formatação de títulos
\usepackage{titling} % Controle da capa
\usepackage{graphicx}
\usepackage{amsmath}
\usepackage{amsfonts}
\usepackage{amssymb}
\usepackage{cite} % Citações ABNT
\usepackage{url}
\usepackage{hyperref}

% Configurações de página
\geometry{
    a4paper,
    left=3cm,
    top=3cm,
    right=2cm,
    bottom=2cm
}

% Espaçamento entre linhas: 1.5
\onehalfspacing

% Configuração de títulos
\titleformat{\section}
    {\normalfont\fontsize{12}{18}\bfseries}
    {\thesection}{1em}{}
    
\titleformat{\subsection}
    {\normalfont\fontsize{12}{18}\bfseries}
    {\thesubsection}{1em}{}

% Remover numeração de páginas na capa
\pagenumbering{gobble}

% ============================================
% CAPA
% ============================================
\begin{document}

\begin{titlepage}
    \centering
    \vspace*{1cm}
    
    {\bfseries PROGRAMA DE PÓS-GRADUAÇÃO EM ENGENHARIA ELÉTRICA\\E DE COMPUTAÇÃO}\\[0.2cm]
    {\bfseries UNIVERSIDADE FEDERAL DE GOIÁS}\\[1cm]
    
    \vspace{4cm}
    
    {\Large\bfseries Agentes Cognitivos Colaborativos para Refinamento Iterativo de Modelos de Decisão em Engenharia Sob Incerteza}\\[2cm]
    
    \vspace{3cm}
    
    {\large Candidato: Werikcyano Lima Guimarães}\\[0.3cm]
    {\large Orientador: Leonardo da Cunha Brito}\\[0.5cm]
    
    \vfill
    
    {\large Goiânia}\\[0.2cm]
    {\large 2026}
    
\end{titlepage}

% Nova página após a capa
\newpage

% A partir daqui, numeração de páginas e sem identificação
\pagenumbering{arabic}
\setcounter{page}{1}

% ============================================
% INTRODUÇÃO E JUSTIFICATIVA
% ============================================
\section{Introdução e Justificativa}

% Conteúdo da introdução será inserido aqui

% ============================================
% OBJETIVOS
% ============================================
\section{Objetivos}

\subsection{Objetivo Geral}

% Objetivo geral será inserido aqui

\subsection{Objetivos Específicos}

\begin{itemize}
    \item Objetivo específico 1
    \item Objetivo específico 2
    \item Objetivo específico 3
    % Adicionar mais objetivos conforme necessário
\end{itemize}

% ============================================
% REFERENCIAL TEÓRICO
% ============================================
\section{Referencial Teórico}

% Conteúdo do referencial teórico será inserido aqui

% ============================================
% METODOLOGIA
% ============================================
\section{Metodologia}

% Conteúdo da metodologia será inserido aqui

% ============================================
% CRONOGRAMA
% ============================================
\section{Cronograma}

% Cronograma será inserido aqui
% Sugestão: usar tabela ou ambiente itemize

% ============================================
% REFERÊNCIAS
% ============================================
\section{Referências}

% Referências serão inseridas aqui usando \cite{} e ambiente thebibliography
% ou arquivo .bib com \bibliography{}

\begin{thebibliography}{99}
    % Exemplo de referência
    % \bibitem{ref1} Autor, A. (Ano). \textit{Título do Trabalho}. Local: Editora.
\end{thebibliography}

\end{document}
