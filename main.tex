\documentclass[12pt,a4paper]{article}

% Configurações de pacotes
\usepackage[utf8]{inputenc}
\usepackage[portuguese]{babel}
\usepackage{times} % Times New Roman
\usepackage{setspace} % Controle de espaçamento
\usepackage{geometry} % Margens
\usepackage{titlesec} % Formatação de títulos
\usepackage{titling} % Controle da capa
\usepackage{graphicx}
\usepackage{amsmath}
\usepackage{amsfonts}
\usepackage{amssymb}
\usepackage[numbers]{natbib} % Citações numéricas (estilo comum em engenharia)
\usepackage{url}
\usepackage{hyperref}
\usepackage{tabularx} % Para tabelas melhores
\usepackage{booktabs} % Para linhas de tabelas profissionais
\usepackage{tikz} % Para esquemáticos se necessário
\usetikzlibrary{shapes,arrows,positioning}

% Configurações de página
\geometry{
    a4paper,
    left=3cm,
    top=3cm,
    right=2cm,
    bottom=2cm
}

% Espaçamento entre linhas: 1.5
\onehalfspacing

% Configuração de títulos
\titleformat{\section}
    {\normalfont\fontsize{12}{18}\bfseries}
    {\thesection}{1em}{}
    
\titleformat{\subsection}
    {\normalfont\fontsize{12}{18}\bfseries}
    {\thesubsection}{1em}{}

% Remover numeração de páginas na capa
\pagenumbering{gobble}

% ============================================
% CAPA
% ============================================
\begin{document}

\begin{titlepage}
    \centering
    \vspace*{1cm}
    
    {\bfseries PROGRAMA DE PÓS-GRADUAÇÃO EM ENGENHARIA ELÉTRICA\\E DE COMPUTAÇÃO}\\[0.2cm]
    {\bfseries UNIVERSIDADE FEDERAL DE GOIÁS}\\[1cm]
    
    \vspace{4cm}
    
    {\Large\bfseries Agentes Cognitivos Colaborativos para Refinamento Iterativo de Modelos de Decisão em Engenharia Sob Incerteza}\\[2cm]
    
    \vspace{3cm}
    
    {\large Candidato: Werikcyano Lima Guimarães}\\[0.3cm]
    {\large Orientador: Leonardo da Cunha Brito}\\[0.5cm]
    
    \vfill
    
    {\large Goiânia}\\[0.2cm]
    {\large 2026}
    
\end{titlepage}

% Nova página após a capa
\newpage

% A partir daqui, numeração de páginas e sem identificação
\pagenumbering{arabic}
\setcounter{page}{1}

% ============================================
% INTRODUÇÃO E JUSTIFICATIVA
% ============================================
\section{Introdução e Justificativa}

A complexidade inerente aos sistemas de engenharia contemporâneos, particularmente no âmbito do setor elétrico e de energia, demanda o emprego de modelos de otimização sofisticados para subsidiar processos decisórios sob condições de incerteza. Tradicionalmente, a transposição de requisitos operacionais para formulações matemáticas rigorosas constitui uma atividade de alta especialização técnica. Observa-se, com frequência, uma dissociação entre o conhecimento prático dos engenheiros de domínio e a formalização necessária para a utilização de ferramentas de pesquisa operatória. Tal cenário é agravado pelo fato de que parcela significativa das restrições fundamentais para a integridade dos sistemas é de natureza tácita, fundamentada na experiência empírica dos especialistas e raramente explicitada em documentos de requisitos formais.

A ascensão dos Modelos de Linguagem de Grande Escala (LLMs) propiciou avanços notáveis na automação da modelagem matemática. Propostas recentes evidenciam a capacidade dessas ferramentas em atuar como interfaces entre a linguagem natural e solvers de otimização, promovendo ganhos em interpretabilidade e agilidade \citep{bbe2f0a73e8bac09457ea17d3b6276dc97f170df}. Adicionalmente, investigações sobre o uso de LLMs como mecanismos de otimização direta demonstram potencial para a exploração de espaços de solução complexos \citep{f8a2dca1e8fe56e698984c077f7ff58d8ca867e9}. Contudo, a aplicação dessas tecnologias em ambientes de engenharia crítica permanece limitada por desafios estruturais, notadamente a propensão à geração de informações factualmente incorretas e a dificuldade de validação semântica frente a protocolos de segurança rigorosos.

A presente investigação propõe a superação de tais obstáculos por meio de uma arquitetura baseada em Sistemas Multi-Agentes (MAS) fundamentada na colaboração entre agentes cognitivos e o especialista humano. A abordagem visa estabelecer um fluxo de refinamento iterativo onde um agente formulador propõe modelos iniciais, enquanto um agente crítico, dotado de bases de conhecimento técnico e ferramentas de verificação formal, atua na identificação de inconsistências e alucinações. 

A justificativa para este trabalho fundamenta-se na necessidade de prover soluções que integrem a automação da modelagem com garantias de confiabilidade e transparência. Ao incorporar metodologias de raciocínio estruturado \citep{1b6e810ce0afd0dd093f789d2b2742d047e316d5} e mecanismos de aprendizado ativo, o arcabouço proposto busca capturar o conhecimento tácito dos engenheiros, transformando a modelagem em um processo de cocriação assistida. Tal proposta alinha-se aos objetivos do programa de pesquisa ao buscar o desenvolvimento de arquiteturas híbridas que priorizem a interpretabilidade e a robustez em sistemas de missão crítica.

% ============================================
% OBJETIVOS
% ============================================
\section{Objetivos}

\subsection{Objetivo Geral}

O objetivo primordial desta pesquisa consiste no desenvolvimento e na avaliação de um arcabouço computacional baseado em agentes cognitivos colaborativos voltado ao refinamento iterativo de modelos de decisão em engenharia. O sistema deve possibilitar a captura de restrições operacionais complexas por meio de interação em linguagem natural, assegurando a viabilidade técnica e a transparência das formulações resultantes.

\subsection{Objetivos Específicos}

\begin{itemize}
    \item Projetar um Agente Formulador fundamentado em modelos de linguagem avançados para a conversão de descrições informais em representações matemáticas formais e código executável para solvers de otimização.
    \item Implementar um Agente Crítico especializado na auditoria de modelos, utilizando heurísticas de engenharia e retroalimentação de solvers para mitigar a ocorrência de alucinações semânticas.
    \item Desenvolver um protocolo de Aprendizado Ativo que identifique ambiguidades nas especificações e interpele o especialista humano para a elucidação de restrições implícitas.
    \item Validar a eficácia da arquitetura proposta por meio de estudos de caso aplicados ao planejamento e operação de sistemas de energia elétrica, comparando-a com métodos convencionais de modelagem.
\end{itemize}

% ============================================
% REFERENCIAL TEÓRICO
% ============================================
\section{Referencial Teórico}

A fundamentação teórica desta pesquisa perpassa pela convergência entre o processamento de linguagem natural e a pesquisa operacional, com ênfase na utilização de modelos generativos para a solução de problemas complexos de engenharia.

\subsection{Evolução dos Modelos de Linguagem e Raciocínio Estruturado}

O desenvolvimento da arquitetura Transformer consolidou um marco na computação ao introduzir mecanismos de atenção que permitem o processamento paralelo de sequências textuais \citep{204e3073870fae3d05bcbc2f6a8e263d9b72e776}. Essa inovação possibilitou o surgimento de modelos com bilhões de parâmetros, capazes de realizar tarefas com mínima necessidade de ajuste fino \citep{90abbc2cf38462b954ae1b772fac9532e2ccd8b0}. Subsequentemente, a introdução de técnicas de estímulo ao raciocínio em etapas, conhecidas como \textit{Chain of Thought}, permitiu que esses modelos decompusessem problemas lógicos complexos em sequências de passos intermediários, elevando significativamente o desempenho em tarefas de inferência e resolução de problemas matemáticos \citep{1b6e810ce0afd0dd093f789d2b2742d047e316d5}.

\subsection{Integração de LLMs na Pesquisa Operacional}

A aplicação de LLMs no domínio da otimização tem evoluído da simples tradução de texto para código em direção à atuação direta como otimizadores. Metodologias baseadas em \textit{Optimization by PROmpting} (OPRO) demonstram que modelos de linguagem podem gerar e refinar soluções para problemas clássicos, como o caixeiro viajante, ao analisar o histórico de desempenhos anteriores inseridos no contexto do prompt \citep{f8a2dca1e8fe56e698984c077f7ff58d8ca867e9}. 

Paralelamente, abordagens que integram algoritmos evolutivos com LLMs, tais como a \textit{Evolution of Heuristics} (EoH) e a \textit{Reflective Evolution} (ReEvo), utilizam a capacidade generativa para conceber novas heurísticas de busca, superando em eficiência métodos projetados manualmente por especialistas \citep{be7a88babf78512b545f585517704cb597388cbc}. Tais avanços evidenciam a viabilidade de sistemas autônomos que não apenas formulam o problema, mas também aprimoram as estratégias de busca de solução.

\subsection{Sistemas de Diálogo e Interpretabilidade em Engenharia}

A aceitação de sistemas baseados em inteligência artificial no ambiente industrial depende intrinsecamente da capacidade de fornecer justificativas compreensíveis para as decisões propostas. Iniciativas como o \textit{OptiChat} e o \textit{OptiMind} focam na criação de sistemas de diálogo que permitem a profissionais sem especialização em otimização interagir com modelos complexos, diagnosticar inviabilidades e realizar análises de sensibilidade em linguagem natural \citep{04eb39fb141f058081036a3c65610c59cd781792, 92b39ce69cc1ee59aff8a92f30557728edc6ea32}. A integração de bibliotecas de experiência, como proposto pelo sistema \textit{AlphaOPT}, fortalece essa tendência ao permitir que o sistema aprenda com sucessos e falhas anteriores, construindo uma base de conhecimento técnico que reduz a incidência de erros semânticos \citep{6f99bcc23e8308650dae64970980bdeae2a4f9d2}.

% ============================================
% METODOLOGIA
% ============================================
\section{Metodologia}

A metodologia proposta para a execução desta pesquisa estrutura-se em fases incrementais, fundamentadas no desenvolvimento e na validação de um sistema multi-agente para suporte à decisão em problemas de engenharia crítica.

\subsection{Arquitetura do Sistema de Agentes Colaborativos}

O arcabouço computacional será composto por dois agentes principais que interagem em um ambiente de refinamento iterativo. O Agente Formulador terá a responsabilidade de interpretar as descrições em linguagem natural fornecidas pelo usuário e convertê-las em modelos de otimização estruturados. Para tal, serão utilizadas técnicas de engenharia de prompt avançadas para garantir a correta definição de variáveis de decisão, funções objetivo e restrições.

O Agente Crítico atuará como um mecanismo de verificação semântica. Esse agente será equipado com uma base de conhecimento técnico específica do setor elétrico e ferramentas de execução de código para validar a sintaxe e a lógica do modelo proposto. Caso sejam detectadas inconsistências ou falhas de viabilidade reportadas pelos solvers de otimização, o Agente Crítico gerará relatórios de erro em linguagem natural para que o Agente Formulador realize os ajustes necessários.

\subsection{Fluxo de Interação Human-in-the-Loop}

A interação com o especialista humano ocorrerá por meio de um módulo de aprendizado ativo. Quando o sistema identificar ambiguidades que impeçam a formalização precisa do problema, interpelará o usuário com questões estruturadas para elucidar o conhecimento tácito. O processo iterativo de crítica e formulação será mantido até que um modelo viável e validado seja obtido. A Figura \ref{fig:fluxo_agentes} ilustra o fluxo de informações proposto para a arquitetura do sistema.

\begin{figure}[ht]
    \centering
    \begin{tikzpicture}[node distance=2cm, auto]
        \tikzstyle{block} = [rectangle, draw, fill=blue!10, text width=5em, text centered, rounded corners, minimum height=4em]
        \tikzstyle{cloud} = [ellipse, draw, fill=red!10, node distance=3cm, minimum height=2em]
        \tikzstyle{line} = [draw, -latex']
        
        \node [cloud] (user) {Engenheiro};
        \node [block, right of=user, node distance=4cm] (form) {Agente Formulador};
        \node [block, below of=form, node distance=3cm] (crit) {Agente Crítico};
        \node [block, left of=crit, node distance=4cm] (solv) {Solvers (Gurobi)};
        
        \path [line] (user) -- node {Requisitos} (form);
        \path [line] (form) -- node {Modelo} (crit);
        \path [line] (crit) -- node {Verificação} (solv);
        \path [line] (solv) -- node {Feedback} (crit);
        \path [line] (crit) -- node {Refinamento} (form);
        \path [line] (crit) -- node {Resultados} (user);
    \end{tikzpicture}
    \caption{Fluxograma simplificado da interação entre os agentes cognitivos e o especialista de domínio.}
    \label{fig:fluxo_agentes}
\end{figure}

\subsection{Cenário de Aplicação e Validação Experimental}

A validação do sistema proposto será realizada por meio de estudos de caso aplicados a problemas reais do setor elétrico brasileiro, como o planejamento da expansão de sistemas de transmissão ou a otimização do despacho de geração renovável. Serão utilizadas ferramentas de modelagem como Pyomo e solvers comerciais de alto desempenho, tais como Gurobi e CPLEX. O desempenho do sistema será avaliado com base na taxa de sucesso na geração de modelos viáveis, no tempo de convergência para o modelo final e na percepção de transparência e confiabilidade reportada por especialistas da área.

% ============================================
% CRONOGRAMA
% ============================================
\section{Cronograma}

O projeto de pesquisa será desenvolvido ao longo de um período de 48 meses, organizado em oito semestres letivos. O planejamento das atividades prioriza o cumprimento dos créditos obrigatórios e a revisão bibliográfica inicial, seguidos pelas etapas de desenvolvimento, experimentação e redação da tese.

\begin{table}[ht]
\centering
\small
\begin{tabularx}{\textwidth}{|X|c|c|c|c|c|c|c|c|}
\hline
\textbf{Atividades / Semestre} & \textbf{1} & \textbf{2} & \textbf{3} & \textbf{4} & \textbf{5} & \textbf{6} & \textbf{7} & \textbf{8} \\ \hline
Créditos em Disciplinas e Revisão Bibliográfica & X & X & X & & & & & \\ \hline
Concepção e Projeto da Arquitetura Multi-Agente & & X & X & & & & & \\ \hline
Desenvolvimento dos Agentes Formulador e Crítico & & & X & X & X & & & \\ \hline
Integração de Módulos de Aprendizado Ativo & & & & X & X & & & \\ \hline
Exame de Qualificação & & & & & X & & & \\ \hline
Realização de Experimentos e Validação & & & & & X & X & X & \\ \hline
Análise de Resultados e Publicações & & & & X & X & X & X & \\ \hline
Redação e Defesa da Tese & & & & & & & X & X \\ \hline
\end{tabularx}
\caption{Cronograma de execução das atividades de doutorado previstas para o período 2026-2030.}
\end{table}

% ============================================
% REFERÊNCIAS
% ============================================
\bibliographystyle{plainnat}
\bibliography{references}

\end{document}
