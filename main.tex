\documentclass[12pt,a4paper]{article}

% Configurações de pacotes
\usepackage[utf8]{inputenc}
\usepackage[portuguese]{babel}
\usepackage{times} % Times New Roman
\usepackage{setspace} % Controle de espaçamento
\usepackage{geometry} % Margens
\usepackage{titlesec} % Formatação de títulos
\usepackage{titling} % Controle da capa
\usepackage{graphicx}
\usepackage{amsmath}
\usepackage{amsfonts}
\usepackage{amssymb}
\usepackage{cite} % Citações ABNT
\usepackage{url}
\usepackage{hyperref}

% Configurações de página
\geometry{
    a4paper,
    left=3cm,
    top=3cm,
    right=2cm,
    bottom=2cm
}

% Espaçamento entre linhas: 1.5
\onehalfspacing

% Configuração de títulos
\titleformat{\section}
    {\normalfont\fontsize{12}{18}\bfseries}
    {\thesection}{1em}{}
    
\titleformat{\subsection}
    {\normalfont\fontsize{12}{18}\bfseries}
    {\thesubsection}{1em}{}

% Remover numeração de páginas na capa
\pagenumbering{gobble}

% ============================================
% CAPA
% ============================================
\begin{document}

\begin{titlepage}
    \centering
    \vspace*{1cm}
    
    {\bfseries PROGRAMA DE PÓS-GRADUAÇÃO EM ENGENHARIA ELÉTRICA\\E DE COMPUTAÇÃO}\\[0.2cm]
    {\bfseries UNIVERSIDADE FEDERAL DE GOIÁS}\\[1cm]
    
    \vspace{4cm}
    
    {\Large\bfseries Agentes Cognitivos Colaborativos para Refinamento Iterativo de Modelos de Decisão em Engenharia Sob Incerteza}\\[2cm]
    
    \vspace{3cm}
    
    {\large Candidato: Werikcyano Lima Guimarães}\\[0.3cm]
    {\large Orientador: Leonardo da Cunha Brito}\\[0.5cm]
    
    \vfill
    
    {\large Goiânia}\\[0.2cm]
    {\large 2026}
    
\end{titlepage}

% Nova página após a capa
\newpage

% A partir daqui, numeração de páginas e sem identificação
\pagenumbering{arabic}
\setcounter{page}{1}

% ============================================
% INTRODUÇÃO E JUSTIFICATIVA
% ============================================
\section{Introdução e Justificativa}

A tomada de decisão em engenharia, especialmente em setores críticos como o de energia, envolve a resolução de problemas de otimização complexos sob incerteza. Tradicionalmente, a formulação desses modelos matemáticos exige um profundo conhecimento técnico, criando uma barreira entre os especialistas de domínio (engenheiros) e as ferramentas de otimização. Além disso, uma parte significativa do conhecimento necessário para uma modelagem robusta é tácita — regras não escritas, intuições baseadas em experiência e restrições operacionais sutis que raramente são capturadas em especificações iniciais.

O advento dos Modelos de Linguagem de Grande Escala (LLMs) introduziu novas possibilidades para a automação dessa modelagem. Trabalhos recentes, como o \textit{OptiGuide} \cite{bbe2f0a73e8bac09457ea17d3b6276dc97f170df}, demonstram como LLMs podem atuar como intérpretes entre a linguagem natural e solvers de otimização, aumentando a interpretabilidade e acessibilidade. Da mesma forma, Yang et al. \cite{f8a2dca1e8fe56e698984c077f7ff58d8ca867e9} exploram o uso de LLMs diretamente como otimizadores. No entanto, a aplicação direta dessas ferramentas em cenários de engenharia crítica enfrenta desafios substanciais: a tendência dos modelos a alucinações, a geração de código sintaticamente correto mas semanticamente inviável, e a dificuldade em validar o modelo gerado contra normas de segurança estritas.

A presente proposta de doutorado busca mitigar essas limitações através de uma abordagem \textit{Human-in-the-loop} suportada por um sistema de Agentes Cognitivos Colaborativos. Propõe-se uma arquitetura onde um agente "Formulador" é responsável pela criação inicial do modelo, enquanto um agente "Crítico" — equipado com ferramentas de verificação e bases de conhecimento de engenharia — atua para validar, refutar e refinar iterativamente as propostas.

A justificativa para esta pesquisa reside na necessidade premente de ferramentas que não apenas automatizem a criação de modelos, mas que garantam sua confiabilidade e alinhamento com a realidade operacional. Ao integrar técnicas de \textit{Chain of Thought} \cite{1b6e810ce0afd0dd093f789d2b2742d047e316d5} e processos de \textit{Active Learning}, o sistema proposto será capaz de identificar ambiguidades e solicitar ativamente o *feedback* do engenheiro humano, transformando o processo de modelagem em um diálogo colaborativo e iterativo. Isso se alinha diretamente com a área de "Aprendizagem de Máquina Explicável e Otimização em Engenharia", promovendo soluções que são transparentes, auditáveis e robustas.

% ============================================
% OBJETIVOS
% ============================================
% ============================================
% OBJETIVOS
% ============================================
\section{Objetivos}

\subsection{Objetivo Geral}

Desenvolver um arcabouço computacional baseado em Sistemas Multi-Agentes (MAS) onde agentes cognitivos colaborativos atuam no refino iterativo de modelos de decisão para engenharia. O sistema deve interagir em linguagem natural com o operador humano, capturando restrições tácitas e validando a viabilidade técnica das soluções propostas através de um processo de crítica e formulação automatizada.

\subsection{Objetivos Específicos}

\begin{itemize}
    \item \textbf{Desenvolver um Agente Formulador:} Projetar e implementar um agente baseado em LLM capaz de traduzir descrições de problemas em linguagem natural para representações formais de otimização (e.g., código Python/Pyomo ou formulações matemáticas).
    \item \textbf{Implementar um Agente Crítico:} Criar um agente especializado na verificação de consistência e viabilidade, utilizando regras de engenharia pré-definidas e feedback de solvers externos para identificar falhas ou alucinações na formulação inicial.
    \item \textbf{Integrar Mecanismo de Aprendizado Ativo:} Desenvolver um módulo de \textit{Active Learning} que permita ao sistema detectar ambiguidades na especificação do problema e formular perguntas estratégicas ao usuário humano para elucidar restrições ocultas.
    \item \textbf{Validar em Estudo de Caso de Energia:} Aplicar e avaliar o sistema proposto em um cenário realista do setor elétrico, como o planejamento da expansão da transmissão ou a operação energética, comparando a eficácia da modelagem assistida por agentes com abordagens manuais tradicionais.
\end{itemize}

% ============================================
% REFERENCIAL TEÓRICO
% ============================================
\section{Referencial Teórico}

% Conteúdo do referencial teórico será inserido aqui

% ============================================
% METODOLOGIA
% ============================================
\section{Metodologia}

% Conteúdo da metodologia será inserido aqui

% ============================================
% CRONOGRAMA
% ============================================
\section{Cronograma}

% Cronograma será inserido aqui
% Sugestão: usar tabela ou ambiente itemize

% ============================================
% REFERÊNCIAS
% ============================================
\section{Referências}

% Referências serão inseridas aqui usando \cite{} e ambiente thebibliography
% ou arquivo .bib com \bibliography{}

\begin{thebibliography}{99}
    
    \bibitem{bbe2f0a73e8bac09457ea17d3b6276dc97f170df}
    Li, B., Mellou, K., Zhang, B., Pathuri, J., Menache, I. (2023). 
    \textit{Large Language Models for Supply Chain Optimization}. 
    ArXiv abs/2307.03875.

    \bibitem{f8a2dca1e8fe56e698984c077f7ff58d8ca867e9}
    Yang, C., Wang, X., Lu, Y., Liu, H., Le, Q. V., Zhou, D., Chen, X. (2023). 
    \textit{Large Language Models as Optimizers}. 
    ArXiv abs/2309.03409.

    \bibitem{1b6e810ce0afd0dd093f789d2b2742d047e316d5}
    Wei, J., Wang, X., Schuurmans, D., Bosma, M., Chi, E. H., Xia, F., ... \& Zhou, D. (2022). 
    \textit{Chain of Thought Prompting Elicits Reasoning in Large Language Models}. 
    ArXiv abs/2201.11903.

    \bibitem{1c259361caa85c2d95a7d04e5e42fa98693da85b}
    Liu, S., Chen, C., Qu, X., Tang, K., Ong, Y. (2023). 
    \textit{Large Language Models as Evolutionary Optimizers}. 
    IEEE Congress on Evolutionary Computation (CEC).

    \bibitem{2fe6060ced80c1c245a718e6188b6516207bf0a8}
    Ramamonjison, R., Li, H., Yu, T. T. L., He, S., Rengan, V., Banitalebi-Dehkordi, A., ... \& Zhang, Y. (2022). 
    \textit{Augmenting Operations Research with Auto-Formulation of Optimization Models from Problem Descriptions}. 
    ArXiv abs/2209.15565.

\end{thebibliography}

\end{document}
